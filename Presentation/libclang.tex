\documentclass[12pt]{beamer}

\usepackage[latin2]{inputenc}
\usepackage{url}
\usepackage{cmbright}
\usepackage{tgadventor}
%\renewcommand*\familydefault{\sfdefault} %% Only if the base font of the document is to be sans serif
\usepackage[T1]{fontenc}
\usepackage{xcolor}
\usepackage{pgf}
\usepackage{tikz}
%\usetikzlibrary{arrows,shapes,chains,positioning,shadows}

\colorlet{darkblue}{blue!75!black}
\colorlet{darkred}{red!75!black}
\colorlet{darkgreen}{green!75!black}
\usepackage{listings}
\lstset{language=C++,
        basicstyle=\ttfamily\footnotesize,
        keywordstyle=\color{darkblue}\ttfamily,
        stringstyle=\color{darkred}\ttfamily,
        commentstyle=\color{gray}\ttfamily,
        morecomment=[l][\color{darkgreen}]{\#},
        numbers=left,
        numbersep=4pt,
        numberstyle=\scriptsize\color{gray},
        keywordsprefix=clang_
}

\usetheme{Madrid}
%\usecolortheme[rgb={0.6,0.1,0.3}]{structure}

\title{libclang: on compiler territory}
\author{Micha� Bartkowiak}
\date{February 13, 2014}

\begin{document}

{
\usebackgroundtemplate{
\vbox to \paperheight{\vfil\hbox to \paperwidth{\hfil
\tikz\node[opacity=0.5] {
\includegraphics[height=\textheight]{1000px-LLVM_Logo.png}};
    \hfil}\vfil}
}%

\begin{frame}
    \titlepage
\end{frame}
}


\section*{Outline}

\usebackgroundtemplate{
\vbox to \paperheight{\vfil\hbox to \paperwidth{\hfil
\tikz\node[opacity=0.1] {
\includegraphics[height=\textheight]{1000px-LLVM_Logo.png}};
    \hfil}\vfil}
}%


\begin{frame}
    \frametitle{Outline}
    \tableofcontents
\end{frame}


\section{Introduction}

\begin{frame}
    \frametitle{What is libclang?}
    \textbf{libclang is a library for processing source code}

    \vspace{2ex}
    \begin{itemize}
        \item Source code translation to Abstract Syntax Trees (AST)
        \item Diagnostic reporting
        \item Traversing AST with cursors
        \item Mapping between cursors and source code
        \item Cross-referencing in the AST
        \item Code completion
        \item Macro expansion
        \item Token extraction and manipulation
    \end{itemize}
\end{frame}


\begin{frame}
    \frametitle{Why libclang?}
    \begin{itemize}
        \item Widely-used and thus verified
        \item Broadest range of parsing capabilities
        \item Simple C API
        \item Detailed information about source code locations at any moment
        \item clang is now trendy ;-)
    \end{itemize}
\end{frame}


\begin{frame}[fragile]
    \frametitle{Where shall we begin?}
    Common header:
\begin{lstlisting}[numbers=none]
#include <clang-c/Index.h>
\end{lstlisting}
\vspace{2ex}
    Create shared index and translation unit:
\begin{lstlisting}[numbers=none]
auto index = clang_createIndex(0, 0);
auto tu = clang_parseTranslationUnit(
              m_index, 0, argv, argc, 0, 0,
              CXTranslationUnit_None);

clang_disposeTranslationUnit(tu);
clang_disposeIndex(index);
\end{lstlisting}
\end{frame}


\begin{frame}[fragile]
    \frametitle{Compilation Flags}
    We would like to have means for generation and storing compilation flags

    \vspace{2ex}
    Solution:\\
    \textbf{JSON Compilation Database Format Specification}

    \vspace{2ex}
    \begin{itemize}
        \item Well defined, portable format
        \item Decouples tools from build systems
        \item Supported systems:
            \begin{itemize}
                \item CMake: via \verb+CMAKE_EXPORT_COMPILE_COMMANDS+ option
                \item Build EAR: via \verb+bear -- make+
            \end{itemize}
        \item libclang can use these flags
    \end{itemize}
\end{frame}


\begin{frame}[fragile]
    \frametitle{JSON Compilation Database - example}
    \begin{scriptsize}
    \begin{verbatim}
[
{
  "directory": "/home/miszak/build/libclang-tools/Apps",
  "command": "/usr/bin/clang++
              -std=c++11 -Wall -Wextra -pedantic -fsanitize=address
              -I/home/miszak/build/libclang-tools/clang+llvm-3.4-x86_64-linux...
              -I/home/miszak/libclang-tools
              -o CMakeFiles/diagnose.dir/Diagnose.cpp.o
              -c /home/miszak/libclang-tools/Apps/Diagnose.cpp",
  "file": "/home/miszak/libclang-tools/Apps/Diagnose.cpp"
},
{
  "directory": "/home/miszak/build/libclang-tools/Apps",
  "command": "/usr/bin/clang++
              -std=c++11 -Wall -Wextra -pedantic -fsanitize=address
              -I/home/miszak/build/libclang-tools/clang+llvm-3.4-x86_64-linux...
              -I/home/miszak/libclang-tools
              -o CMakeFiles/function_name_check.dir/FunctionNameCheck.cpp.o
              -c /home/miszak/libclang-tools/Apps/FunctionNameCheck.cpp",
  "file": "/home/miszak/libclang-tools/Apps/FunctionNameCheck.cpp"
}
]
    \end{verbatim}
\end{scriptsize}
\end{frame}


\section{Diagnostics}

\begin{frame}[fragile]
    \frametitle{Obtaining Diagnostics}
Given the translation unit \verb+tu+:
\vspace{1ex}
\begin{lstlisting}[numbers=none]
for (auto diagNum : clang_getNumDiagnostics(tu))
{
    auto diag = clang_getDiagnostic(tu, diagNum);
    auto diagStr =
        clang_formatDiagnostic(diag,
            clang_defaultDiagnosticDisplayOptions());

    std::cout << clang_getCString(diagStr) << std::endl;

    clang_disposeString(diagStr);
}
\end{lstlisting}
\end{frame}


\begin{frame}[fragile]
    \frametitle{Diagnostics - Example}
    From:
\begin{lstlisting}
class X
{
    const int a;
}
\end{lstlisting}
    we will get \emph{formatted} output:
    \begin{itemize}
        \item class.cpp:1:7: warning: class 'X' does not declare any constructor to initialize its non-modifiable members
        \item class.cpp:4:2: error: expected ';' after class
        \item class.cpp:3:15: warning: private field 'a' is not used [-Wunused-private-field]
    \end{itemize}
\end{frame}


\begin{frame}[fragile]
    \frametitle{Diagnostics - Details}
    Each information about diagnostic can be obtained separately:
    \begin{itemize}
        \item clang\_getDiagnosticSeverity
        \item clang\_getDiagnosticSpelling
        \item clang\_getDiagnosticLocation and clang\_getSpellingLocation
        \item clang\_getDiagnosticNumRanges and clang\_getDiagnosticRange
    \end{itemize}
    \vspace{2ex}
    But we want more\ldots
\end{frame}


\begin{frame}[fragile]
    \frametitle{Diagnostics - Fix-its}
\begin{lstlisting}[numbers=none, basicstyle=\ttfamily\scriptsize]
for (auto fixitNum: clang_getDiagnosticNumFixIts(diag))
{
    CXSourceRange range;
    auto fixItStr =
        clang_getDiagnosticFixIt(diag, fixitNum, &range);

    auto rangeStart = clang_getRangeStart(range);
    auto rangeEnd = clang_getRangeEnd(range);

    unsigned lStart, cStart, lEnd, cEnd;
    clang_getSpellingLocation(
            rangeStart, 0, &lStart, &cStart, 0);
    clang_getSpellingLocation(
            rangeEnd, 0, &lEnd, &cEnd, 0);

    std::cout << lStart << ":" << cStart << " - " <<
              << lEnd   << ":" << lEnd << ": " <<
              clang_getCString(fixItStr) << std::endl;

    clang_disposeString(fixItStr);
}
\end{lstlisting}
\end{frame}


\begin{frame}[fragile]
    \frametitle{Diagnostics - Fix-its - Output}
\begin{columns}
  \begin{column}{0.45\textwidth}
    As simple as:
    \vspace{1ex}

    \texttt{4:2 - 4:2: ;}\\

    \vspace{3ex}
    In line \textbf{4}, \\  in column \textbf{2} \\ put \textbf{;}
  \end{column}
  \begin{column}{0.45\textwidth}
\begin{lstlisting}[keywordsprefix=_]
class X
{
    const int a;
}_
\end{lstlisting}
  \end{column}
\end{columns}
\end{frame}


\section{Walking the Abstract Syntax Tree}

\begin{frame}
    \frametitle{Walking the AST with CXCursor}
    CXCursor represents generalised AST node

    \vspace{1ex}
    It can represent e.g.:
    \begin{itemize}
        \item declaration
        \item definition
        \item statement
        \item reference
    \end{itemize}

    \vspace{1ex}
    Provides:
    \begin{itemize}
        \item name
        \item location and range in source code
        \item type information
        \item child(ren)
    \end{itemize}
\end{frame}


\begin{frame}[fragile]
    \frametitle{Learning to Walk}
    It is simple!

    \vspace{2ex}
    Provide:
\begin{lstlisting}[numbers=none]
typedef enum CXChildVisitResult (* CXCursorVisitor)(
        CXCursor cursor,
        CXCursor parent,
        CXClientData client_data)
\end{lstlisting}

    \vspace{1ex}
    and use:
\begin{lstlisting}[numbers=none]
unsigned clang_visitChildren(
        CXCursor parent,
        CXCursorVisitor visitor,
        CXClientData client_data)
\end{lstlisting}
\end{frame}


\begin{frame}[fragile]
    \frametitle{First Visit: Guest}
\begin{lstlisting}[numbers=none]
CXChildVisitResult guest(
    CXCursor cursor,
    CXCursor parent,
    CXClientData client_data)
{
    switch (clang_getCursorKind(cursor))
    {
        case CXCursor_FunctionDecl:
            std::cout << "function"; break;
        case CXCursor_CXXMethod:
            std::cout << "cxxmethod"; break;
        default:
            std::cout << "other"; break;
    }
    std::cout << std::endl;
    return CXChildVisit_Recurse;
}
\end{lstlisting}
\end{frame}


\begin{frame}[fragile]
    \frametitle{First Visit}
\begin{columns}
\begin{column}{0.65\textwidth}
\begin{lstlisting}[numbers=none]
unsigned clang_visitChildren(
    clang_getTranslationUnitCursor(tu),
    guest, 0)
\end{lstlisting}

\vspace{2ex}
Example:
\begin{lstlisting}
void f1();
namespace A
{
    void f2();
    class Y
    {
        void m1() {};
    };
    template<typename T> T ft1();
}
\end{lstlisting}
\end{column}
\begin{column}{0.2\textwidth}

\vspace{6ex}

Output:
\begin{verbatim}
function
other
function
other
cxxmethod
other
other
other
other
\end{verbatim}
\end{column}
\end{columns}
\end{frame}


\begin{frame}[fragile]
    \frametitle{When Things Get More Complicated}
    Example was trivial.

    \vspace{1ex}
    What to do when translation unit has (many) includes?
    \vspace{1ex}

\begin{lstlisting}[numbers=none]
auto sourceLoc = clang_getCursorLocation(cursor);
CXFile file;
clang_getFileLocation(sourceLoc, &file, 0, 0, 0);
auto fileName = clang_getFileName(file);

// skip cursors which are not in our file
if (fileName != "/path/to/our/file.cpp")
{
    return CXChildVisit_Continue;
}
\end{lstlisting}

We can \emph{always} learn CXCursor's detailed location.
\end{frame}


\begin{frame}[fragile]
    \frametitle{What About Parents?}
Given the cursor, we can learn about two kinds of parents:
\begin{itemize}
    \item lexical: \verb+clang_getCursorLexicalParent+
    \item semantic: \verb+clang_getCursorSemanticParent+
\end{itemize}

\vspace{1ex}
\begin{lstlisting}
namespace N
{
    class C
    {
        void foo();
    };

    void C::foo() { /* ... */ }
}
\end{lstlisting}

\vspace{1ex}
For declarations: \verb+clang_getCursorDefinition+
\end{frame}


\begin{frame}[fragile]
    \frametitle{Reference Cursors}
If the cursor kind is \verb+CXCursor_*Ref+ (Type, Variable\ldots),\\
then we can learn about the referenced entity: \\
\verb+clang_getCursorReferenced+

\vspace{2ex}
This way we can find all local references to type, variable\ldots
And we are able to e.g.:
\begin{itemize}
    \item rename them (refactoring)
    \item colour them (\emph{semantic} highlighting)
    \item jump between occurences
    \item jump between reference and declaration
\end{itemize}
\end{frame}


\begin{frame}[fragile]
    \frametitle{Unified Symbol Resolutions}
Each \verb+CXCursor+ with external linkage\\ can be uniquely identified by
USR:

\vspace{1ex}
\verb+clang_getCursorUSR+

\vspace{3ex}
This way we can deal with declarations\\ \emph{across} translation units.

\vspace{3ex}
Example: \verb+c:@N@A@C@X@F@m1#+ \\
for \verb+A::Y::m1+ (method \verb+m1+ in class \verb+X+ in namespace \verb+A+)
\end{frame}


\begin{frame}[fragile]
    \frametitle{Tokens}
Cursors enable us to see the code from AST perspective

\vspace{1ex}
Sometimes we just want tokens, e.g. in \emph{syntax highlighting}

\begin{lstlisting}[numbers=none]
clang_tokenize(
        CXTranslationUnit TU,
        CXSourceRange Range,
        CXToken **Tokens,
        unsigned *NumTokens)
\end{lstlisting}

\vspace{1ex}
For each token we can obtain:
\begin{itemize}
    \item kind (\verb+clang_getTokenKind+):\\
          keyword, identifier, punctuation, literal, comment
    \item source location and range (\verb+clang_getTokenLocation+)
    \item spelling (\verb+clang_getTokenSpelling+)
    \item corresponding cursor (\verb+clang_annotateTokens+)
\end{itemize}
\end{frame}


\section{Code Completion}

\begin{frame}[fragile]
    \frametitle{Code Completion}
This is the moment when C-api becomes horryfying\ldots

\vspace{1ex}
\begin{lstlisting}[numbers=none]
clang_codeCompleteAt(
        CXTranslationUnit tu,
        const char * complete_filename,
        unsigned complete_line,
        unsigned complete_column,
        struct CXUnsavedFile * unsaved_files,
        unsigned num_unsaved_files,
        unsigned options)
\end{lstlisting}

\end{frame}


\begin{frame}[fragile]
    \frametitle{Code Completion: Example}
\begin{lstlisting}[numbers=none,basicstyle=\ttfamily\scriptsize]
auto compls = clang_codeCompleteAt(
                    tu, "fileName.cpp", 13, 7, 0, 0,
                    clang_defaultCodeCompleteOptions());

for (auto i = 0u; i < compls->NumResults; ++i)
{
  auto &complStr = completionResults->Results[i].CompletionString;

  for (auto j = 0u; j < clang_getNumCompletionChunks(complStr); ++j)
  {
    auto chunkStr = clang_getCompletionChunkText(complStr, j);
    std::cout << toString(chunkStr) << "_";
  }

  std::cout << std::endl;
}

clang_disposeCodeCompleteResults(compls);
\end{lstlisting}

\vspace{1ex}
\begin{scriptsize}
* A bit of clang\_dispose* function calls is omitted...
\end{scriptsize}
\end{frame}


\begin{frame}[fragile]
    \frametitle{Code Completion: Example}
\begin{columns}
\begin{column}{0.50\textwidth}
\begin{lstlisting}
class A
{
    void fp() {};
public:
    void f1() {};
    void f2(int k = 0) {};
    void f3(int i) {};
};

void foo()
{
    auto a = A();
    a.
}
\end{lstlisting}
\end{column}
\begin{column}{0.44\textwidth}
\begin{footnotesize}
\begin{verbatim}
void_f2_(__)_
void_f3_(_int i_)_
void_~A_(_)_
A &_operator=_(_const A &_)_
A &_operator=_(_A &&_)_
A_::_
void_fp_(_)_
void_f1_(_)_
\end{verbatim}
\end{footnotesize}
\end{column}
\end{columns}
\end{frame}


\begin{frame}
    \frametitle{Code Completion: Algorithm}
Client triggers completion procedure at proper place\\ (e.g. at "."
when it follows class/struct instance) and presents initial suggestions

\vspace{2ex}
The starting place is remembered

\vspace{2ex}
Then following procedure is done for each newly typed character:
\begin{itemize}
    \item trigger code completion
    \item filter the results basing on contents of token
    \item present suggestions
\end{itemize}
\end{frame}


\begin{frame}[fragile]
    \frametitle{Code Completion: Even More}
For each completion we can also:
\begin{itemize}
    \item obtain its priority (\verb+clang_getCompletionPriority+)
    \item get its context(s) (\verb+clang_codeCompleteGetContexts+)
    \item for container context get kind of the container
        (\verb+clang_codeCompleteGetContainerKind+)
    \item obtain brief comment (\verb+lang_getCompletionBriefComment+)
\end{itemize}
\end{frame}


\section{Tools}

\begin{frame}[fragile]
    \frametitle{c-index-test}
    Use \textbf{c-index-test} for experiments

\vspace{1ex}
\begin{scriptsize}
\begin{verbatim}
usage: c-index-test -code-completion-at=<site> <compiler arguments>
       c-index-test -code-completion-timing=<site> <compiler arguments>
       c-index-test -cursor-at=<site> <compiler arguments>
       [...]

$ c-index-test -code-completion-at=<filename>:13:7 <filename + args>
\end{verbatim}
\end{scriptsize}

\vspace{2ex}
Output:
\begin{scriptsize}
\begin{verbatim}
ClassDecl:{TypedText A}{Text ::} (75)
CXXMethod:{ResultType void}{TypedText f1}{LeftParen (}{RightParen )} (34)
CXXMethod:{ResultType void}{TypedText f2}{LeftParen (}{Optional {Placeholder int k}}
    {RightParen )} (34)
[...]
\end{verbatim}
\end{scriptsize}
\end{frame}


\begin{frame}[fragile]
    \frametitle{libclang in Python}
    Want to use libclang capabilities in Python? \\
    Not a problem

\vspace{1ex}
\begin{itemize}
    \item clang.cindex module: copy it or set PYTHONPATH
        (warning: Python bindings are part of clang's source)
    \item clang.cindex needs to be able to find the libclang.so
    \item \verb+import clang.cindex+ \\
          \verb+index = clang.cindex.Index.create()+ \\
          \verb+tu = index.parse(sys.argv[1])+
\end{itemize}

\vspace{1ex}
More on this:
\url{http://eli.thegreenplace.net/2011/07/03/parsing-c-in-python-with-clang/}
\end{frame}


\section{What's Next?}

\begin{frame}
    \frametitle{What's Next?}
\textbf{Create awesome developer tools!}
\vspace{1ex}
\begin{itemize}
    \item improvements for IDEs
    \item code completion and syntax checking available for virtually any
          text editor (e.g. Vim ;-))
    \item refactoring tools
    \item static code analyzers
\end{itemize}

\vspace{2ex}
If C api is too clumsy dive directly into clang's C++ interface \\
(and make presentation about it!)
\end{frame}


\section{References}

\begin{frame}[fragile]
    \frametitle{References}
    \begin{footnotesize}
    \begin{thebibliography}{9}
        \bibitem{1} \url{http://clang.llvm.org/doxygen/}
        \bibitem{2} \url{http://clang.llvm.org/docs/Tooling.html}
        \bibitem{3} \url{http://llvm.org/devmtg/2010-11/Gregor-libclang.pdf}
        \bibitem{4} \url{http://eli.thegreenplace.net/2011/07/03/parsing-c-in-python-with-clang/}
        \bibitem{5} \url{http://llvm.org/devmtg/2011-11/Gregor_ExtendingClang.pdf}
        \bibitem{6} \url{https://github.com/llvm-mirror/clang/tree/master/tools/c-index-test}
        \bibitem{7} \url{https://github.com/miszak/libclang-tools}
        \bibitem{8} \url{https://github.com/Valloric/YouCompleteMe}
        \bibitem{9} \url{https://github.com/axw/cmonster}
    \end{thebibliography}
\end{footnotesize}
\end{frame}


{
\usebackgroundtemplate{
\vbox to \paperheight{\vfil\hbox to \paperwidth{\hfil
\tikz\node[opacity=0.5] {
\includegraphics[height=\textheight]{1000px-LLVM_Logo.png}};
    \hfil}\vfil}
}%

\begin{frame}
    \titlepage
\end{frame}
}


\end{document}

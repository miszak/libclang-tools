\documentclass[12pt]{beamer}

\usepackage[latin2]{inputenc}
\usepackage{url}
%\usepackage{pgf}
%\usepackage{tikz}
%\usetikzlibrary{arrows,shapes,chains,positioning,shadows}

\usetheme{Madrid}
%\usecolortheme[rgb={0.6,0.1,0.3}]{structure}

\title{libclang: on compiler territory}
\author{Micha� Bartkowiak}
\date{\today}

\begin{document}

\begin{frame}
    \titlepage
\end{frame}


\section*{Outline}
\begin{frame}
    \frametitle{Outline}
    \tableofcontents
\end{frame}


\section{Why libclang?}
\begin{frame}
    \frametitle{Why libclang?}
    \begin{itemize}
        \item Widely-used and thus verified
        \item Broadest range of parsing capabilities
        \item ...
    \end{itemize}
\end{frame}

\section{libclang in Python}
\begin{frame}
    \frametitle{libclang in Python}
    Want to use libclang capabilities in Python? Not a problem.
\end{frame}

\subsection{What's Next?}
\begin{frame}
    \frametitle{What's Next}
    \begin{itemize}
        \item ...
    \end{itemize}
\end{frame}

\section{References}

\begin{frame}[fragile]
    \frametitle{References}
    \begin{thebibliography}{9}
        \bibitem{1} \url{some url}
    \end{thebibliography}
\end{frame}


\begin{frame}
    \titlepage
\end{frame}


\end{document}
